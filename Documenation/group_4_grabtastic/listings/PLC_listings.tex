 \lstset{
   basicstyle=\scriptsize\ttfamily,
   keywordstyle=\bfseries\ttfamily\color{orange},
   stringstyle=\color{green}\ttfamily,
   commentstyle=\color{middlegray}\ttfamily,
   emph={square}, 
   emphstyle=\color{blue}\texttt,
   emph={[2]root,base},
   emphstyle={[2]\color{yac}\texttt},
   showstringspaces=false,
   flexiblecolumns=false,
   tabsize=2,
   numbers=left,
   numberstyle=\tiny,
   numberblanklines=false,
   stepnumber=1,
   numbersep=11pt,
   xleftmargin=15pt,
   breaklines=true
	language=Pascal}
	
\subsection{Allgemein}
\begin{lstlisting}[language=pascal, captionpos=t, caption=Allgemein]

(*safe the min and max value of the force*)
IF maxForceValuePos < acutalForceValue THEN
	maxForceValuePos := acutalForceValue;
END_IF
IF maxForceValueNeg > acutalForceValue THEN
	maxForceValueNeg := acutalForceValue;
END_IF

\end{lstlisting}


\subsection{Main}
\begin{lstlisting}[language=pascal, captionpos=t, caption=Main]
VAR CONSTANT 
	BufStart:INT:=  1;      // Start-Index
 	BufEnd:INT:=  100;   // End-Index
END_VAR

VAR
	diX, diY AT %I* : BOOL;
	(*boole variables*)
	bStart :BOOL := FALSE;
	bMoveHome : BOOL := FALSE;
	bDauerlauf : BOOL := FALSE;
	i: INT;
	btest : BOOL;
	btestZwei : bool;
	btesDrei: BOOL;
END_VAR

posAxisX(Axis:= axisX, Enable:= TRUE , Valid=> , Busy=> , Error=> , ErrorID=> , Position=> );
posAxisY(Axis:= axisY, Enable:= TRUE, Valid=> , Busy=> , Error=> , ErrorID=> , Position=> );

powerX(Axis:= axisX, Enable:= bEnable, Enable_Positive:= bEnable, Enable_Negative:= bEnable, Override:= 100, BufferMode:= , Options:= , Status=> , Busy=> , Active=> , Error=> , ErrorID=> );
powerY(Axis:= axisY, Enable:= bEnable, Enable_Positive:= bEnable, Enable_Negative:= bEnable, Override:= 100, BufferMode:= , Options:= , Status=> , Busy=> , Active=> , Error=> , ErrorID=> );

resetX(Axis:= axisX, Execute:= , Done=> , Busy=> , Error=> , ErrorID=> );
resetY(Axis:= axisY, Execute:= , Done=> , Busy=> , Error=> , ErrorID=> );

homeX(Axis:= axisX, Execute:= , Position:= 0 , HomingMode:= , BufferMode:= , Options:= , bCalibrationCam:= NOT diX, Done=> , Busy=> , Active=> , CommandAborted=> , Error=> , ErrorID=> );
homeY(Axis:= axisY, Execute:= , Position:= 0 , HomingMode:= , BufferMode:= , Options:= , bCalibrationCam:= NOT diY, Done=> , Busy=> , Active=> , CommandAborted=> , Error=> , ErrorID=> );	

moveAbsX(Axis:= axisX, Execute:= , Position:= , Velocity:= , Acceleration:= , Deceleration:= , Jerk:= , BufferMode:= , Options:= , Done=> , Busy=> , Active=> , CommandAborted=> , Error=> , ErrorID=> );

moveAbsY(Axis:= axisY, Execute:= , Position:= , Velocity:= , Acceleration:= , Deceleration:= , Jerk:= , BufferMode:= buffermoveAbsY, Options:= , Done=> , Busy=> , Active=> , CommandAborted=> , Error=> , ErrorID=> );

moveAbsNewY(Axis:= axisY, Execute:= , Position:= , Velocity:= , Acceleration:= , Deceleration:= , Jerk:= , BufferMode:= buffermoveAbsY, Options:= , Done=> , Busy=> , Active=> , CommandAborted=> , Error=> , ErrorID=> );	

SCL(Crtl:= , bSend:= , ReceivedString=> , bNewString=> , bSendBusy=> , bStringReceived=> );
	
	rTrigStartCycle(CLK:= bStart , Q=> );
	CASE iState OF
		
	0:	IF (*rTrig.Q AND *)powerX.Status AND powerY.Status THEN
			iState := iState + 1;
		END_IF

	1:	SCL.Crtl := 'I';
		SCL.bSend := TRUE;
		respondOfORder := '';
		istate := 2;

	2:	IF NOT SCL.bSendBusy THEN
			SCL.bSend := FALSE;
			istate := 3;
		END_IF

	3:	IF waitToLongForRespond.Q THEN
			istate := 1;
		END_IF
		IF respondI.Q1 THEN
			respondI.RESET := TRUE;
			respondI.SET1 := FALSE;
			istate := 5;
		END_IF	
	
	5:		resetX.Execute :=TRUE;
			resetY.Execute :=TRUE;
			iState := 10;
		
	10:	IF resetX.Done AND resetY.Done THEN
			resetX.Execute := FALSE;
			resetY.Execute := FALSE;
			iState :=  iState + 5;
		END_IF
		
	15: IF NOT readInfoAxisX.Status.Homed or NOT readInfoAxisY.Status.Homed THEN
		homeX.Execute := TRUE;
		homeY.Execute := TRUE;
		iState := 20;
		ELSE
			iState := 21; //--> if already referenced, go home position
		END_IF
	20:	IF homeX.Done AND homeY.Done THEN
			homeX.Execute := FALSE;
			homeY.Execute := FALSE;
			NotHomed.RESET := FALSE;
			iState := 21;		
		END_IF
	21:			moveAbsY.Velocity := posHomeVelocityY;
				moveAbsY.Position := posHomeY;
				moveAbsY.Execute := TRUE;

			IF posAxisY.Position > posPrepositioningY THEN
				moveAbsX.Velocity := posHomeVelocityX;
				moveAbsX.Position := posHomeX;
				moveAbsX.Execute := TRUE;
				iState :=  25;
			END_IF

	25:	IF 	moveAbsX.Done AND moveAbsY.Done THEN
			moveAbsX.Execute:= FALSE;
			moveAbsY.Execute:= FALSE;
			respondH.RESET := TRUE;
			respondH.SET1 := FALSE;
			iState :=  26;
		END_IF
		
	(*this starts a new cycle, depending on the load*)
	26:	IF rTrigStartCycle.Q OR bDauerlauf THEN
			IF bLoaded THEN
				iState :=  125;
			ELSE
				iState :=  27;
			END_IF
			respondD.RESET := TRUE;
			respondD.SET1 := FALSE;
			respondH.RESET := TRUE;
			respondH.SET1 := FALSE;
			respondP.RESET := TRUE;
			respondP.SET1 := FALSE;
			cycleFinished := FALSE;
		END_IF				
		
	27:		moveAbsX.Velocity := posRepositoryVelocityX;
			moveAbsX.Position := posRepositoryX;
			moveAbsX.Execute := TRUE;
			moveAbsY.Velocity := posRepositoryVelocityY;
			moveAbsY.Position :=posRepositoryY;
			moveAbsY.Execute := TRUE;
			iState :=  28;
			maxForceValuePos := 0;
			maxForceValueNeg := 0;

	28: 	SCL.Crtl := 'P'; // vorpositionieren
			SCL.bSend := TRUE;
			iState :=  29;
		
	29:	IF NOT SCL.bSendBusy THEN
			SCL.bSend := FALSE;
			SCL.Crtl := '';
			istate := 30;
		END_IF
		
	30:	IF waitToLongForRespond.Q THEN
			istate := 28;
			END_IF
		IF 	moveAbsX.Done AND moveAbsY.Done AND respondP.Q1 THEN
			moveAbsX.Execute:= FALSE;
			moveAbsY.Execute:= FALSE;
			respondP.RESET := TRUE;
			respondP.SET1 := FALSE;
			iState := 31;
		END_IF
		
	31:	SCL.Crtl := 'D';
		SCL.bSend := TRUE;
		istate := 32;
		
	32:	IF NOT SCL.bSendBusy THEN
			SCL.bSend := FALSE;
			SCL.Crtl := '';
			istate := 33;
		END_IF

	33:	IF waitToLongForRespond.Q THEN
			istate := 31;
			END_IF
		IF respondD.Q1 THEN //Move Down
			respondD.RESET := TRUE;
			respondD.SET1 := FALSE;
			bLoaded := TRUE;
			cycleFinished :=TRUE;
			istate := 35;
		END_IF	
		
	35:		IF NOT respondD.Q1 THEN
				moveAbsY.Velocity := posHomeVelocityY;
				moveAbsY.Position :=posHomeY;
				moveAbsY.Execute := TRUE;
			IF posAxisY.Position > posPrepositioningY THEN //über dem weg drausen, dan darf er fahren
				moveAbsX.Velocity := posHomeVelocityX;
				moveAbsX.Position := posHomeX;
				moveAbsX.Execute := TRUE;
				iState :=  40;
			END_IF
		
		END_IF
	40:	IF 	moveAbsX.Done AND moveAbsY.Done THEN
			moveAbsX.Execute:= FALSE;
			moveAbsY.Execute:= FALSE;
			iState := 26;
		END_IF
		
	125:	moveAbsX.Velocity := 5;
			moveAbsX.Position := posPrepositioningX;
			moveAbsX.Execute := TRUE;
			moveAbsY.Velocity := posPrepositioningVelocityY;
			moveAbsY.Position :=posPrepositioningY;
			moveAbsY.Execute := TRUE;
			iState :=  130;
			maxForceValuePos := 0;
			maxForceValueNeg := 0;
			
	130:	IF moveAbsX.Done THEN
				moveAbsNewY.Execute:= TRUE;
				moveAbsNewY.Velocity := posRepositoryVelocityY;
				moveAbsNewY.Position := posRepositoryY;
				moveAbsY.Execute:= FALSE;
				moveAbsX.Execute:= FALSE;
			END_IF
			IF moveAbsNewY.Done THEN
				iState := 140;
		END_IF
		
	(*135:	moveAbsX.Velocity := 	posRepositoryVelocityX;
			moveAbsX.Position := 	posRepositoryX;
			moveAbsX.Execute := 	TRUE;
			moveAbsY.Velocity :=	posRepositoryVelocityY;
			moveAbsY.Position :=	posRepositoryY;
			moveAbsY.Execute := TRUE;
			iState :=  140;*)
			
	140:	IF 	(*moveAbsX.Done AND*) moveAbsNewY.Done THEN
			moveAbsNewY.Execute:= FALSE;
			iState :=  141;
			END_IF
			
	141:	SCL.Crtl := 'P'; //Move Up
			SCL.bSend := TRUE;
			istate := 142;
			
	142:	IF NOT SCL.bSendBusy THEN
			SCL.bSend := FALSE;
			SCL.Crtl := '';
			istate := 143;
		END_IF

	143:	IF waitToLongForRespond.Q THEN
				istate := 141;
			END_IF
			IF respondP.Q1 THEN
				respondP.RESET := TRUE;
				respondP.SET1 := FALSE;
				bLoaded := FALSE;
				cycleFinished :=TRUE;
				iState := 145;
			END_IF	
			
	145:	SCL.Crtl := 'H'; // Home pos anfahren
			SCL.bSend := TRUE;
			iState :=  150;
		
	150:	IF NOT SCL.bSendBusy THEN
				SCL.bSend := FALSE;
				SCL.Crtl := '';
				istate := 155;
			END_IF

 	155:	moveAbsY.Velocity := posHomeVelocityY;
				moveAbsY.Position := posHomeY;
				moveAbsY.Execute := TRUE;

			IF posAxisY.Position > posPrepositioningY THEN
				moveAbsX.Velocity := posHomeVelocityX;
				moveAbsX.Position := posHomeX;
				moveAbsX.Execute := TRUE;
				iState :=  160;
			END_IF
		
	160:	IF waitToLongForRespond.Q THEN
				istate := 145;
			END_IF
			IF 	moveAbsX.Done AND moveAbsY.Done AND respondH.Q1 THEN
			moveAbsX.Execute:= FALSE;
			moveAbsY.Execute:= FALSE;
			respondH.RESET := TRUE;
			respondH.SET1 := FALSE;
			iState :=  26;
		END_IF
		
	END_CASE
	
	resetErrorHandling();
	visu();
\end{lstlisting}
\subsection{Error and Reset}

\begin{lstlisting}[language=pascal, captionpos=t, caption=Error and Reset]
VAR
	reset: INT := 0; //reset state machine
END_VAR

VAR CONSTANT 
	toMuchForcePositive : INT := 15000;
	toMuchForceNegative : INT := -15000;
END_VAR

respondD(SET1:= , RESET:= , Q1=> );
respondH(SET1:= , RESET:= , Q1=> );
respondI(SET1:= , RESET:= , Q1=> );
respondP(SET1:= , RESET:= , Q1=> );
respondOk(SET1:= , RESET:= , Q1=> );
respondStartetUp(SET1:= , RESET:= , Q1=> );
emergencyStopToMuchForce(SET1:= , RESET:= , Q1=> );
motorXYNoPower(SET1:= , RESET:= , Q1=> );
errorNoSignalFromUc(SET1:= , RESET:= , Q1=> );
unequalLoadThanExpected(SET1:= , RESET:= , Q1=> );
zAxisNotMoving(SET1:= , RESET:= , Q1=> );

rTrigStartViewPressed(CLK:= bStartViewPressed , Q=> );

IF respondI.Q1 THEN
		respondI.SET1 := FALSE;
END_IF

respondD.RESET := FALSE;
respondH.RESET := FALSE;
respondI.RESET := FALSE;
respondP.RESET := FALSE;
respondOK.RESET := FALSE;
respondStartetUp.RESET := FALSE;
emergencyStopToMuchForce.RESET := FALSE;
motorXYNoPower.RESET := FALSE;
errorNoSignalFromUc.RESET := FALSE;
unequalLoadThanExpected.RESET := FALSE;
zAxisNotMoving.RESET:=FALSE;

(*read status from x and y*)
readInfoAxisX(Axis:= axisX, Enable:= TRUE, Valid=> , Busy=> , Error=> , ErrorID=> , ErrorStop=> , Disabled=> , Stopping=> , StandStill=> , DiscreteMotion=> , ContinuousMotion=> , SynchronizedMotion=> , Homing=> , ConstantVelocity=> , Accelerating=> , Decelerating=> , Status=> );
readInfoAxisY(Axis:= axisY, Enable:= TRUE, Valid=> , Busy=> , Error=> , ErrorID=> , ErrorStop=> , Disabled=> , Stopping=> , StandStill=> , DiscreteMotion=> , ContinuousMotion=> , SynchronizedMotion=> , Homing=> , ConstantVelocity=> , Accelerating=> , Decelerating=> , Status=> );

(* wait to seconds to give the stepper motor drivers a little time to built DC link*)
waitWithEnable(IN:= bEmergencyPressed, PT:= T#2S, Q=> , ET=> );

(* if no respond from the uC in a little time, the nsend command again*)
waitToLongForRespond(IN:= 	iState = 33  OR iState = 143 OR iState =3 OR iState = 30, PT:= T#10S, Q=> , ET=> );

zAxisNotMoving.SET1:= (globalErrorFromUC = '2E');

(* Ungleich soll beladung *)
IF (iState = 23 OR iState = 0) AND ((acutalForceValue > -200 AND bLoaded) OR (acutalForceValue < -700 AND NOT bLoaded))THEN
	unequalLoadThanExpected.SET1:=TRUE;
END_IF

(* error if no respond from uC*)
errorNoSignalFromUc.SET1 := NOT noSiganlFromU.Q;

(* trigger after restart from emergency *)
fTrigReset(CLK:= waitWithEnable.Q, Q=> );

(* set if we ar not homed*)
NotHomed(SET1:= NOT readInfoAxisX.Status.Homed OR NOT readInfoAxisY.Status.Homed, RESET:= , Q1=> );
motorXYNoPower.SET1 :=  readInfoAxisY.ErrorStop OR readInfoAxisX.ErrorStop;
(* Status not ok but Emergency is ok*)
globalErrorDetect := motorXYNoPower.Q1 OR respondStartetUp.Q1 OR emergencyStopToMuchForce.Q1 OR bEmergencyPressed OR errorNoSignalFromUc.Q1 OR unequalLoadThanExpected.Q1 OR zAxisNotMoving.Q1;
(* set error *)
setError(SET1:=globalErrorDetect, RESET:= , Q1=> );

CASE reset OF
0:	IF fTrigReset.Q OR ((setError.Q1 OR NotHomed.SET1) AND (bReset OR rTrigStartViewPressed.Q)) THEN
		reset := 5;
		iState := 0;
		bEnable := FALSE;
		globalErrorDetect := FALSE;
		setError.RESET :=TRUE;
		respondStartetUp.RESET := TRUE;
		bLoaded := FALSE;
		respondStartetUp.SET1 := FALSE;
		emergencyStopToMuchForce.RESET := TRUE;
		emergencyStopToMuchForce.SET1 := FALSE;
		motorXYNoPower.SET1 := FALSE;
		motorXYNoPower.RESET := TRUE;
		errorNoSignalFromUc.SET1 := FALSE;
		errorNoSignalFromUc.RESET := TRUE;
		unequalLoadThanExpected.SET1:= FALSE;
		unequalLoadThanExpected.RESET := TRUE;
		zAxisNotMoving.SET1 := FALSE;
		zAxisNotMoving.RESET := TRUE;
		respondI.RESET := TRUE;
		respondI.SET1 := FALSE;
		respondD.RESET := TRUE;
		respondD.SET1 := FALSE;
		respondH.RESET := TRUE;
		respondH.SET1 := FALSE;
		maxForceValuePos := 0;
		maxForceValueNeg := 0;
	END_IF
5:	resetX.Execute :=TRUE;
	resetY.Execute :=TRUE;
	setError.RESET :=FALSE;
	reset := 10;
	
10:	IF resetX.Done AND resetY.Done THEN
		resetX.Execute := FALSE;
		resetY.Execute := FALSE;
		reset :=  15;
	END_IF
15: bEnable := TRUE;
	reset  := 0;
END_CASE

(* if to much force --> emergency stop!*)
IF (acutalForceValue > toMuchForcePositive) OR (acutalForceValue < toMuchForceNegative) THEN
	emergencyStopToMuchForce.SET1 := TRUE;
END_IF



(*reset all movements and sending*)
IF setError.Q1 THEN
	iState := 0;
	homeX.Execute := FALSE;
	homeY.Execute := FALSE;
	moveAbsX.Execute:= FALSE;
	moveAbsY.Execute:= FALSE;
	moveAbsNewY.Execute:= FALSE;
	bEnable := FALSE;
	SCL.bSend := FALSE;
	SCL.Crtl := '';
END_IF

\begin{lstlisting}[language=pascal]
\end{lstlisting}



\subsection{Serial Com}
\begin{lstlisting}[language=pascal, captionpos=t, caption=Serial Communication]
VAR_INPUT
	Crtl: STRING;
	bSend: BOOL;
END_VAR
VAR_OUTPUT
	ReceivedString: STRING;
	bNewString: BOOL;
	bSendBusy: BOOL;
	bStringReceived: BOOL;
END_VAR
(*VAR_IN_OUT
	IOrespondOfORder: STRING;
END_VAR*)
VAR
	//DataIn AT %I*: EL6inData22B;
	//DataOut AT %Q*: EL6outData22B;
	SerialLineCtrl: SerialLineControl;
	//BuffTX: ComBuffer;
	//BuffRX: ComBuffer;
	clearbuff: ClearComBuffer;
	SendString: SendString;
	RecString: ReceiveString;
	istate: INT;
	rtrig: r_trig;
	test: ReceiveByte;
	bTest: BYTE;
	bByteReceived : BOOL := FALSE;
	bBuffer: BYTE;
	arrForceValues : ARRAY [1..1000] OF INT;

	lengtOfReceivedString: INT;
	strTemp : STRING(255);
	posSuffix: INT;
	stest :STRING;
	
	posFirstSlash: INT;
	posSecondSlash: INT;
	posThirdSlash: INT;
	posHash: INT;
END_VAR


SendString(SendString:= , Busy=> , Error=> , TXbuffer:= BuffTX);
RecString(Prefix:= , Suffix:= '$N', Timeout:= T#2S, Reset:= , StringReceived=> , 	Busy=> , 	Error=> , 	RxTimeout=> , 	ReceivedString:= ReceivedString, 	RXbuffer:= BuffRX);
posHash := FIND (ReceivedString,'#');
posSuffix := FIND (ReceivedString,'$R');
posFirstSlash := FIND (ReceivedString,'/');
strTemp := MID (ReceivedString,posSuffix - posFirstSlash-1,posFirstSlash+1);
posSecondSlash := FIND (strTemp,'/') + posFirstSlash;
strTemp := MID (ReceivedString,posSuffix - posSecondSlash-1,posSecondSlash+1);
posThirdSlash := FIND (strTemp,'/') + posSecondSlash;

lengtOfReceivedString:=LEN(STR:= ReceivedString);

IF RecString.StringReceived THEN
	strTemp := MID (ReceivedString,posSecondSlash - posHash-1,posHash+1);
	acutalForceValue := STRING_TO_INT (strTemp);
	
	strTemp := MID (ReceivedString,posThirdSlash - posSecondSlash-1,posSecondSlash+1);
	posAxisZ := STRING_TO_INT (strTemp);

	globalErrorFromUC := MID (ReceivedString,posSuffix - posThirdSlash-1,posThirdSlash+1);
	
	globalForceValueArrived := TRUE;
	RecString.Reset := TRUE;

	respondOfORder := MID (ReceivedString,posFirstSlash-1,1); (* else its a respond of the Order*)
	RecString.Reset := TRUE;
	respondI.SET1 := respondOfORder = 'i';
	respondH.SET1 := respondOfORder = 'h';
	respondD.SET1 := respondOfORder = 'd';
	respondP.SET1 := respondOfORder = 'p';
	respondOK.SET1 := respondOfORder = 'ok';
	respondStartetUp.SET1 :=  respondOfORder  = 'Started up';
	respondOfORder := '';
	counterSerialCom := counterSerialCom+1;
END_IF


(* no isgnal from uC *)
noSiganlFromU(IN:= RecString.StringReceived, PT:= T#2S, Q=> , ET=> );


rtrig(CLK:= bSend, Q=> );
CASE istate OF

0:	RecString.Reset := FALSE;
	bStringReceived := FALSE;
	IF rtrig.Q THEN
		SendString.SendString := Crtl;
		bSendBusy := TRUE;
		istate := istate  + 1;
	END_IF	

1:	IF NOT SendString.Busy THEN
			SendString.SendString := '';
			(*clearbuff(Buffer:= BuffTX );*)
			istate := istate + 1;
			ReceivedString := '';
			bSendBusy := FALSE;
	END_IF

2:	IF RecString.StringReceived THEN
		bStringReceived := TRUE;
		RecString.Reset := TRUE;
		istate := 0;
	END_IF
		
END_CASE


\end{lstlisting}

\subsection{Visualization}
\begin{lstlisting}[language=pascal, captionpos=t, caption=Visualization]
VAR
	//stringErrorText : WSTRING; //error texts for visu
	//stringMessageText : WSTRING;
	//stringDebugText : WSTRING;
	stringErrorText: ARRAY [1..20] OF STRING;
	stringMessageText: ARRAY [1..20] OF STRING;
	stringDebugText: ARRAY [1..20] OF STRING;
	forceMemoryDrop: writeInArray;
	forceMemoryPick: writeInArray;
	zaehl: INT := 1;
END_VAR

forceMemoryDrop(i_iState:= iState, i_actualForceValue:= acutalForceValue, i_bLoaded:= TRUE, io_ValueArrived:= globalForceValueArrived);
forceMemoryPick(i_iState:= iState, i_actualForceValue:= acutalForceValue, i_bLoaded:= FALSE, io_ValueArrived:= globalForceValueArrived);
//cycleFinished(SET1:= , RESET:= , Q1=> );

(* pos of x and y for the visu*)
IF readInfoAxisX.Status.Homed AND readInfoAxisY.Status.Homed	THEN
realVisuPosX := posAxisX.Position * 1.5;
realVisuPosY := (posAxisY.Position * (-1.0) + 10) *1.7;
//realVisuPosX := ABS (realVisuPosX);
ELSE
realVisuPosX := 0.0;
realVisuPosY := 0.0;	
END_IF

IF bLoaded THEN
	realVisuDiskX := realVisuPosX; (* - posRepositoryX * 1.5;*)
	realVisuDiskY := realVisuPosY; (*- (posRepositoryY * (-1.0) + 10) *2.0;	*)
ELSE
	realVisuDiskX := posRepositoryX * 1.5;
	realVisuDiskY := (posRepositoryY * (-1.0) + 10) *2.0;
END_IF

IF iState > 3 THEN
intVisuPosZ := (posAxisZ) *2;
ELSE
intVisuPosZ := 0;	
END_IF

(*prepare cycle finished*)

(* visu texts *)

MEMSET(ADR(stringErrorText), 0, SIZEOF(stringErrorText));
MEMSET(ADR(stringMessageText), 0, SIZEOF(stringMessageText));
MEMSET(ADR(stringDebugText), 0, SIZEOF(stringDebugText));

zaehl := 1;
(* error messages *)
IF NOT readInfoAxisX.Status.Homed OR NOT readInfoAxisY.Status.Homed THEN
	stringErrorText[zaehl]:= CONCAT(STR1:= stringErrorText[zaehl], STR2:= 'Not referenced in X/Y');
	zaehl := zaehl + 1;
END_IF

IF  globalErrorDetect THEN
	stringErrorText[zaehl]:= CONCAT(STR1:= stringErrorText[zaehl], STR2:= 'Failure');
	zaehl := zaehl + 1;
END_IF

IF  bEmergencyPressed THEN
	stringErrorText[zaehl]:= CONCAT(STR1:= stringErrorText[zaehl], STR2:= 'Emergency Stop');
	zaehl := zaehl + 1;
END_IF

IF  emergencyStopToMuchForce.Q1 THEN
	stringErrorText[zaehl]:= CONCAT(STR1:= stringErrorText[zaehl], STR2:= 'too much Force');
	zaehl := zaehl + 1;
END_IF

IF motorXYNoPower.Q1 THEN
	stringErrorText[zaehl]:= CONCAT(STR1:= stringErrorText[zaehl], STR2:= 'Not referenced in X/Y');	
	zaehl := zaehl + 1;
END_IF

IF errorNoSignalFromUc.Q1 THEN
	stringErrorText[zaehl]:= CONCAT(STR1:= stringErrorText[zaehl], STR2:= 'error connection lost to uC');
	zaehl := zaehl + 1;
END_IF

IF unequalLoadThanExpected.Q1 THEN
	stringErrorText[zaehl]:= CONCAT(STR1:= stringErrorText[zaehl], STR2:= 'load unequal than expected');	
	zaehl := zaehl + 1;
END_IF

IF zAxisNotMoving.Q1 THEN
	stringErrorText[zaehl]:= CONCAT(STR1:= stringErrorText[zaehl], STR2:= 'z-axis Error');	
	zaehl := zaehl + 1;
END_IF

(* massage textes *)
	zaehl := 1;
IF acutalForceValue > -200  THEN
	stringMessageText[zaehl]:= CONCAT(STR1:= stringMessageText[zaehl], STR2:= 'The object is successfully dropped');
	zaehl := zaehl + 1;
END_IF

IF acutalForceValue < -700 THEN
	stringMessageText[zaehl]:= CONCAT(STR1:= stringMessageText[zaehl], STR2:= 'The object is successfully grabbed');
	zaehl := zaehl + 1;
END_IF


IF iState = 26 AND cycleFinished THEN
	stringMessageText[zaehl]:= CONCAT(STR1:= stringMessageText[zaehl], STR2:= 'Cycle successfully finished');
	zaehl := zaehl + 1;
END_IF

(* debug messages *)
	zaehl := 1;
IF bLoaded THEN	
	stringDebugText[zaehl]:= CONCAT(STR1:= stringDebugText[zaehl], STR2:= 'Load');
	zaehl := zaehl + 1;
ELSE
	stringDebugText[zaehl]:= CONCAT(STR1:= stringDebugText[zaehl], STR2:= 'No Load');
	zaehl := zaehl + 1;
END_IF

stringDebugText[zaehl]:= CONCAT(STR1:= stringDebugText[zaehl], STR2:= 'State PLC: ');
stringDebugText[zaehl]:= CONCAT(STR1:= stringDebugText[zaehl], STR2:= INT_TO_STRING (iState));
zaehl := zaehl + 1;
stringDebugText[zaehl]:= CONCAT(STR1:= stringDebugText[zaehl], STR2:= 'Error from uC:');
stringDebugText[zaehl]:= CONCAT(STR1:= stringDebugText[zaehl], STR2:= globalErrorFromUC);
zaehl := zaehl + 1;

\end{lstlisting}

\subsection{Write in Array}
\begin{lstlisting}[language=pascal, captionpos=t, caption=Write in Array]
VAR CONSTANT 
	BufStart:INT:=  1;      // Start-Index
 	BufEnd:INT:=  100;   // End-Index
END_VAR

VAR_INPUT
	i_iState : INT;
	i_actualForceValue : INT;
	i_bLoaded : BOOL;
END_VAR
VAR_OUTPUT
	
END_VAR
VAR
	sTemp : INT;
	sArrForceValue: ARRAY [BufStart..BufEnd] OF INT;
END_VAR

VAR_IN_OUT
	io_ValueArrived : BOOL;
END_VAR


IF io_ValueArrived AND ((NOT i_bLoaded AND (iState = 33) AND (posAxisZ > 60 )) OR (i_bLoaded AND (iState = 143) AND (posAxisZ > 57 ))) THEN
		sTemp := sTemp + 1;
		IF sTemp > BufEnd THEN
			sTemp:=1;
		END_IF
		sArrForceValue[sTemp] := i_actualForceValue;
		io_ValueArrived := FALSE;		
	END_IF
	
	IF	(((i_iState = 27 AND NOT i_bLoaded) OR (i_iState = 125 AND i_bLoaded)) ) THEN //delet Buffer Begin from new
		MEMSET(ADR(sArrForceValue), 0, SIZEOF(sArrForceValue));
		sTemp:=1;
	END_IF

\end{lstlisting}